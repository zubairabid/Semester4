\documentclass[11pt,a4paper,twocolumn]{article}
\usepackage{fontspec}
\defaultfontfeatures{Mapping=tex-text}
\usepackage{xunicode}
\usepackage{xltxtra}
%\setmainfont{???}
\usepackage{amsmath}
\usepackage{amsfonts}
\usepackage{amssymb}
\usepackage{graphicx}
\usepackage[left=1cm,right=1cm,top=2cm,bottom=2cm]{geometry}

\usepackage[document]{ragged2e}

\font\serif="FreeSerif:script=beng"
\font\bg="FreeSerif:script=beng" at 13pt

\usepackage{xcolor}
\definecolor{hlit}{rgb}{0.5,0,1}
\definecolor{diff}{rgb}{0.1,0.1,0.5}
\definecolor{rlit}{rgb}{1,0,0}

\title{Computational Linguistics - 2, Graded Exercise 2}
\author{Zubair Abid, 20171076}
\date{}

\begin{document}
	\twocolumn[
	\begin{@twocolumnfalse}
    	\maketitle
    	The three tasks of the exercise:\\
    	\begin{enumerate}
    		\item Comparing Indian POS tagger output on
    		previous assignment data to own output, and
    		analyzing and commenting on the differences.
    		
    		\item Chunk given data with proper labels 
    		(English)
    		
    		\item Chunk Indian language data with proper
    		labels
    	\end{enumerate}
		$ $\\
		$ $\\
    	
  	\end{@twocolumnfalse} 
	]

	\section{Part 1}
	
	\section{Part 2: Case frames for first language}
	{\bg
	\begin{enumerate}
		
		\item আমি কাল বিকেলে খেলছিলাম\\
		\textcolor{hlit}{বিকেলে} -> \textcolor{rlit}{আমি} (Agent)\\
		\textcolor{hlit}{বিকেলে} -> \textcolor{rlit}{কাল বিকেলে} (Time)\\
				
		
		\item সূর্য পূর্ব দিক থেকে ওঠে \\
		\textcolor{hlit}{ওঠে} -> \textcolor{rlit}{সূর্য} (Force)\\
		\textcolor{hlit}{ওঠে} -> \textcolor{rlit}{পূর্ব দিক} (Source)\\
				
		\item দেবজিৎ আর শান্তনু লীগ খেলছিল \\
		\textcolor{hlit}{খেলছিল} -> \textcolor{rlit}{লীগ} (Theme)\\
		\textcolor{hlit}{খেলছিল} -> \textcolor{rlit}{দেবজিৎ আর শান্তনু} (Agent (combined))\\
				
		\item আমার এক্সাম কলকাতায় হবে \\
		\textcolor{hlit}{হবে} -> \textcolor{rlit}{আমার এক্সাম} (Theme)\\
		\textcolor{hlit}{হবে } -> \textcolor{rlit}{কলকাতায়} (Location)\\
				
		\item ডাক্তারটা পেশেন্টের ওপর অপারেশন করলো স্কাল্পেল্ দিয়ে \\
		\textcolor{hlit}{অপারেশন করলো} -> \textcolor{rlit}{স্কাল্পেল্} (Instrument)\\
		\textcolor{hlit}{অপারেশন করলো} -> \textcolor{rlit}{ডাক্তারটা} (Agent)\\
		\textcolor{hlit}{অপারেশন করলো} -> \textcolor{rlit}{পেশেন্ট} (Patient)\\
				
	\end{enumerate}		
	}	
	
	\section{Part 3}
	{\bg
	\begin{enumerate}
	
		\item CONSUME\\
		\textcolor{hlit}{খাওয়া} (to eat), \textcolor{hlit}{পান করা} (to drink), 
		\textcolor{hlit}{নিস্সাস নেওয়া} (to inhale), ()\\
		
		\item ATRANS\\
		\textcolor{hlit}{কেনা} (to buy), \textcolor{hlit}{বিকরি করা} (to sell), 
		\textcolor{hlit}{ধার দেওয়া} (to lend), \textcolor{hlit}{ধার নেওয়া} (to borrow)\\
		
		\item PTRANS\\
		\textcolor{hlit}{হাঁটা} (to walk), \textcolor{hlit}{দৌড়ানো} (to run), 
		\textcolor{hlit}{খুড়ানো} (to limp), \textcolor{hlit}{হামাগুড়ি দেওয়া} (to crawl)\\
		
		\item MTRANS\\
		\textcolor{hlit}{বোঝানো} (to explain), \textcolor{hlit}{কথা বলা} (to talk), 
		\textcolor{hlit}{পরামর্শ দেওয়া} (to advise), ()\\
		
		\item MBUILD\\
		\textcolor{hlit}{ভাবা} (to think), \textcolor{hlit}{শপনো দেখা} (to dream), 
		(), ()\\
		
		\item ATTEND\\
		\textcolor{hlit}{শোনা} (to hear), \textcolor{hlit}{দেখা} (to see), 
		\textcolor{hlit}{ছুঁয়া} (to touch), ()\\
		
		\item SPEAK\\
		\textcolor{hlit}{বলা} (to speak), \textcolor{hlit}{শোনানো} (to make listen), 
		(), ()\\
		
		\item PROPEL\\
		\textcolor{hlit}{ধাক্কা দেওয়া} (to push), \textcolor{hlit}{টানা} (to pull), 
		(), ()\\
		
		\item MOVE\\
		\textcolor{hlit}{মারা} (to hit), \textcolor{hlit}{লাথি মারা} (to kick), 
		\textcolor{hlit}{ঘুষি মারা} (to punch), \textcolor{hlit}{ওঠানো} (to lift)\\
		
		\item EXPEL\\
		\textcolor{hlit}{প্রসাব করা} (to pee), \textcolor{hlit}{পায়খানা করা} (to shit), 
		\textcolor{hlit}{}(), \textcolor{hlit}{}()\\
		
		\item GRASP\\
		\textcolor{hlit}{ধরা} (to hold), \textcolor{hlit}{}(), 
		\textcolor{hlit}{}(), \textcolor{hlit}{}()\\
			
	
	\end{enumerate}
	
	}

\end{document}